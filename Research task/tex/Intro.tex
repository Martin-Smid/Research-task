\section*{Phase 0: Solving the Schrödinger vacuum equation}

Equation:
\begin{equation}
    i \frac{\partial \psi(x,t)}{\partial t} = - \frac{1}{2} \partial_x \psi(x,t)
\end{equation}
Dividing this by $\psi$ and integrating over time we get: 
\begin{equation}
    \int \frac{\partial \psi}{\psi} = \int A \partial t \xrightarrow{\text{which leads to:}} \psi(x,t+h) = \text{e}^{\int A} \psi (x,t) = e^{\int i \frac{1}{2} \partial^2_x \text{d}t}\psi(x,t)
\end{equation}
The exponential term can be expanded using Taylor series:
\begin{equation}
    e^{\int i \frac{1}{2} \partial^2_x \text{d}t} = 1 + \partial^2_x + \frac{\partial^4_x}{2} + \dots
\end{equation}
To which we can apply Furier transform, where derivation simply become multiplication by a parameter, and again using Taylor series we get:
\begin{equation}
    \widetilde{\psi}(x,t+h) = e^{\int i \frac{h}{2} k^2 \text{d}t}\widetilde{\psi}(x,t)
\end{equation}
Then we can simply use inverse Furier transform:
\begin{equation}
    \psi(x,t+h) = \mathscr{F}^{-1} \left[ e^{\int i \frac{h}{2} k^2 \text{d}t}\widetilde{\psi}(x,t) \right] 
\end{equation}

In code for simplicity, initial conditions are chosen to be: 
\begin{itemize}
    \item Boundaries: <a = 0, b = 1>
    \item Weave function: Complex Gausssian \begin{equation}
        \psi_0 (x) = A \cdot e^{-\frac{(x-x_0)^2}{2\sigma^2_0}} e^{ik_0 x}
    \end{equation}
    \item With $x_0 = 0.5$, $\sigma_0 = 0.1$ and initial momenta $k_0 = 0$
\end{itemize}

\section*{Phase 1: Solving the Schrödinger equation dependent on $r^2$}

We include the harmonic potential given by:
\begin{equation}
    V(r) = \frac{1}{2}  \omega^2 r^2 \hspace{0.5cm} \text{with} \hspace{0.5cm} r^2 = x^2 + y^2 + z^2.
\end{equation} 
For simplicity, we begin with a two-dimensional system ($n_x$ and $n_y$), and assume natural units where $\hbar = 1$, $m = 1$, and $\omega = 1$. The equation to solve is as follows:

\begin{equation}
-i \hbar \frac{\partial \Psi}{\partial t} = -\frac{\hbar^2}{2m} \nabla^2 \Psi + \frac{1}{2}m \omega^2 (x^2 + y^2 + z^2)\Psi,
\end{equation}


\subsection*{Analytical Solution}

The analytical solution for this system can be expressed in terms of Hermite polynomials and Gaussian functions. The stationary wave functions are as follows:

\begin{equation}
\psi_s(x, y, z) = C(n_x, n_y, n_z) H_{n_x}(x) H_{n_y}(y) H_{n_z}(z) 
\times e^{-\frac{\beta^2 (x^2 + y^2 + z^2)}{2}},
\end{equation}


where $H_{n}(x)$ are Hermite polynomials of order $n$, and $C_{n_x, n_y,n_z}$ is a normalization constant given by:

\begin{equation}
C(n_x, n_y, n_z) = \left(\frac{\beta^2}{\pi}\right)^{3/4} \frac{1}{\sqrt{2^{n_x + n_y + n_z} n_x! n_y! n_z!}},
\end{equation}
\begin{equation}
\beta = \sqrt{\frac{m \omega}{\hbar}}.
\end{equation}




\subsection*{Numerical Approach}

We solve the Schrödinger equation using a split-step Fourier method. The approach involves alternating between applying the kinetic energy operator in Fourier space and the potential energy operator in real space. The evolution of the wave function over a small time step $h$ is given by:

\begin{equation}
\psi(x,y,t+h) = e^{-i \frac{h}{2} \nabla^2} e^{-i h V(x,y)} e^{-i \frac{h}{2} \nabla^2} \psi(x,y,t).
\end{equation}

\subsubsection*{Initial Conditions}

For now the work is done on a 2D Gaussian wave packet:

\begin{equation}
\psi_0(x, y) = A \cdot e^{-\frac{(x-x_0)^2 + (y-y_0)^2}{2\sigma^2}} e^{i(k_x x + k_y y)},
\end{equation}

with $x_0 = 0.5$, $y_0 = 0.5$, $\sigma = 0.1$, and initial momenta $k_x = 0$, $k_y = 0$. The boundaries of the domain are set to $x, y \in [0, 1]$.

\subsubsection*{Error Analysis}

To assess the accuracy of the numerical solution, two types of errors must be monitored:

Error in the Wavefunction:
\begin{equation}
\text{Err}{\psi} = \left| \psi_{\text{numerical}}(x,y,t) - \psi_{\text{analytical}}(x,y) \right|.
\end{equation}
This measures the deviation of the numerical solution from the analytical stationary solution.

Error in the Probability Density:
\begin{equation}
\text{Err}{\rho} = \left| \left| \psi_{\text{numerical}}(x,y,t) \right|^2 - \left| \psi_{\text{numerical}}(x,y,t=0) \right|^2 \right|.
\end{equation}
As the analytical solution is stationary, the probability density $|\psi|^2$ should remain constant over time. Any deviation provides a measure of the numerical error.



